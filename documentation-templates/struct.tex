\documentclass[11pt, letterpaper]{article}\usepackage[]{graphicx}\usepackage[]{color}
% maxwidth is the original width if it is less than linewidth
% otherwise use linewidth (to make sure the graphics do not exceed the margin)
\makeatletter
\def\maxwidth{ %
  \ifdim\Gin@nat@width>\linewidth
    \linewidth
  \else
    \Gin@nat@width
  \fi
}
\makeatother

\definecolor{fgcolor}{rgb}{0.345, 0.345, 0.345}
\newcommand{\hlnum}[1]{\textcolor[rgb]{0.686,0.059,0.569}{#1}}%
\newcommand{\hlstr}[1]{\textcolor[rgb]{0.192,0.494,0.8}{#1}}%
\newcommand{\hlcom}[1]{\textcolor[rgb]{0.678,0.584,0.686}{\textit{#1}}}%
\newcommand{\hlopt}[1]{\textcolor[rgb]{0,0,0}{#1}}%
\newcommand{\hlstd}[1]{\textcolor[rgb]{0.345,0.345,0.345}{#1}}%
\newcommand{\hlkwa}[1]{\textcolor[rgb]{0.161,0.373,0.58}{\textbf{#1}}}%
\newcommand{\hlkwb}[1]{\textcolor[rgb]{0.69,0.353,0.396}{#1}}%
\newcommand{\hlkwc}[1]{\textcolor[rgb]{0.333,0.667,0.333}{#1}}%
\newcommand{\hlkwd}[1]{\textcolor[rgb]{0.737,0.353,0.396}{\textbf{#1}}}%
\let\hlipl\hlkwb

\usepackage{framed}
\makeatletter
\newenvironment{kframe}{%
 \def\at@end@of@kframe{}%
 \ifinner\ifhmode%
  \def\at@end@of@kframe{\end{minipage}}%
  \begin{minipage}{\columnwidth}%
 \fi\fi%
 \def\FrameCommand##1{\hskip\@totalleftmargin \hskip-\fboxsep
 \colorbox{shadecolor}{##1}\hskip-\fboxsep
     % There is no \\@totalrightmargin, so:
     \hskip-\linewidth \hskip-\@totalleftmargin \hskip\columnwidth}%
 \MakeFramed {\advance\hsize-\width
   \@totalleftmargin\z@ \linewidth\hsize
   \@setminipage}}%
 {\par\unskip\endMakeFramed%
 \at@end@of@kframe}
\makeatother

\definecolor{shadecolor}{rgb}{.97, .97, .97}
\definecolor{messagecolor}{rgb}{0, 0, 0}
\definecolor{warningcolor}{rgb}{1, 0, 1}
\definecolor{errorcolor}{rgb}{1, 0, 0}
\newenvironment{knitrout}{}{} % an empty environment to be redefined in TeX

\usepackage{alltt}
\usepackage[utf8]{inputenc}
\usepackage{amsmath}
\usepackage{xcolor}
\usepackage{geometry}
\usepackage[parfill]{parskip}
\usepackage{float}
\usepackage{graphicx}
\usepackage{fancyhdr}
\usepackage{appendix}
\graphicspath{{D:/OneDrive/__Projects/econforecasting/documentation-templates/images/}}
\newcommand{\vv}[1]{\textcolor{black}{\mathbf{#1}}}
\definecolor{econgreen}{RGB}{55, 91, 1}

\geometry{left=2.0cm, right = 2.0cm, top = 3cm, bottom = 3cm}

\fancypagestyle{plain}{
	\let\oldheadrule\headrule% Copy \headrule into \oldheadrule
	\renewcommand{\headrule}{\color{econgreen}\oldheadrule}
	\lhead{\small{\textcolor{black}{\leftmark}}}
	%\chead{}
	\rhead{\small{\textcolor{black}{\thepage}}}
	\lfoot{}
	\cfoot{}
	\rfoot{\textit{charles@cmefi.com}}
	\renewcommand{\headrulewidth}{0.5pt}
	%\renewcommand{\footrulewidth}{0.5pt}
}\pagestyle{plain}
\IfFileExists{upquote.sty}{\usepackage{upquote}}{}
\begin{document}
%%%%%%%%%%%%%%%%%%%%%%%%%%%%%%%%%%%%%%%%%%%%%%%%%%%%%%%%%%%%%%%%%%%%%
\begin{titlepage}
\thispagestyle{empty}
\newgeometry{left=5cm, top=5cm} %defines the geometry for the titlepage
\pagecolor{econgreen}
\noindent
\includegraphics[width=2cm]{cmefi_short.png} \\[-1em]
\color{white}
\makebox[0pt][l]{\rule{1.3\textwidth}{1pt}}
\par
\noindent
%\textbf{\textsf{A Macroeconomic Nowcasting Model}} 
%\vskip5cm
{\Huge \textsf{A Nowcasting Model for Time Series with Ragged-Edge Data}}
\vskip\baselineskip
\noindent
\textsf{Model Run Date: June 25, 2021}\\
\textsf{charles@cmefi.com}
\restoregeometry % restores the geometry
\nopagecolor% Use this to restore the color pages to white
\end{titlepage}
%%%%%%%%%%%%%%%%%%%%%%%%%%%%%%%%%%%%%%%%%%%%%%%%%%%%%%%%%%%%%%%%%%%%%%




\tableofcontents
\newpage
\listoftables
\listoffigures
\newpage
\section{Central Structural Model}
\subsection{Equations}

Consider the following structural equations:
\begin{align*}
    dlog(\vv{gdp}_t) &= \frac{\vv{pce}_{ss}}{\vv{gdp}_{ss}} dlog(\vv{pce}_t) + \frac{\vv{pdi}_{ss}}{\vv{gdp}_{ss}} dlog(\vv{pdi}_t) + \frac{\vv{im}_{ss}}{\vv{gdp}_{ss}} dlog(\vv{im}_t) -\\
    &\quad{} \frac{\vv{ex}_{ss}}{\vv{gdp}_{ss}} dlog(\vv{ex}_t) + \frac{\vv{govt}_{ss}}{\vv{gdp}_{ss}} dlog(\vv{govt}_t)\\
   	dlog(\vv{pce_t}) &= \widehat{\beta_0} + \widehat{\beta_1} \frac{1}{4} \sum_{j=0}^1 dlog(\vv{dpi}_{t-j}) + e_t\\
   	dlog(\vv{pdi_t}) &= \widehat{\beta_0}  + \widehat{\beta_1} \vv{pdi}_{t-1} + e_t\\
   	dlog(\vv{govt_t}) &=  \widehat{\beta_0}  + \widehat{\beta_1} \vv{govt}_{t-1} + e_t\\
   	dlog(\vv{ex_t}) &=  \widehat{\beta_0}  + \widehat{\beta_1} \vv{ex}_{t-1} + e_t\\
   	dlog(\vv{im_t}) &=  \widehat{\beta_0}  + \widehat{\beta_1} \vv{im}_{t-1} + e_t
\end{align*}


We can move this into matrix form:
\setcounter{MaxMatrixCols}{20}
\begin{align*}
\begin{bmatrix}
	0\\
	\widehat{\beta^2_0}\\
	\widehat{\beta^3_0}\\
	\widehat{\beta^4_0}\\
	\widehat{\beta^5_0}\\
	\widehat{\beta^6_0}
\end{bmatrix}
=
\begin{bmatrix}
	1 & -\frac{pce}{gdp} & -\frac{pdi}{gdp} & -\frac{im}{gdp} & \frac{ex}{gdp} & -\frac{govt}{gdp} & 0 & 0 & 0 & 0 & 0 & 0 & 0\\
	0 & 1 & 0 & 0 & 0 & 0 & 0 & 0 & 0 & 0 & 0 & -\widehat{\beta^2_1} & -\widehat{\beta^2_1}\\
	0 & 0 & 0 & 0 & 0 & 0 & 0 & 0 & 0 & 0 & 0 & 0 & 0
\end{bmatrix}
\begin{bmatrix}
	gdp_{t}\\
	pce_{t}\\
	pdi_{t}\\
	im_{t}\\
	ex_{t}\\
	govt_{t}\\
	pce_{t-1}\\
	pdi_{t-1}\\
	im_{t-1}\\
	ex_{t-1}\\
	govt_{t-1}\\
	dpi_{t}\\
	dpi_{t-1}
\end{bmatrix}
\end{align*}
%\begin{align*}
%\underbrace{\begin{bmatrix}
%		gdp_{t}\\
%		pce_{t}\\
%		pdi_{t}\\
%		govt_{t}\\
%		ex_{t}\\
%		im_{t}
%\end{bmatrix}}_{z_t}
%=
%\begin{bmatrix}
%	0 & \frac{\vv{pce}_{ss}}{\vv{gdp}_{ss}} & \frac{\vv{pdi}_{ss}}{\vv{gdp}_{ss}} & \frac{\vv{im}_{ss}}{\vv{gdp}_{ss}} & -\frac{\vv{ex}_{ss}}{\vv{gdp}_{ss}} & \frac{\vv{govt}_{ss}}{\vv{gdp}_{ss}}\\
%	0 & 0 & 0 & 0 & 0 & 0
%\end{bmatrix}
%\underbrace{\begin{bmatrix}
%		gdp_{t-1}\\
%		pce_{t-1}\\
%		pdi_{t-1}\\
%		govt_{t-1}\\
%		ex_{t-1}\\
%		im_{t-1}
%\end{bmatrix}}_{z_{t-1}}
%+
%\begin{bmatrix}
%	0 & 0\\
%	\widehat{\beta^2_0} & 0\\
%	\widehat{\beta^3_0} & \widehat{\beta^3_1}\\
%	\widehat{\beta^4_0} & 0\\
%	\widehat{\beta^5_0} & 0\\
%	\widehat{\beta^6_0} & 0
%\end{bmatrix}
%\begin{bmatrix}
%	1\\
%	\vv{dpi}_t
%\end{bmatrix}
%\end{align*}



\begin{align*}
    dlog(ue_t) = 
\end{align*}

The next step is to model the transition of the factors over time. To do so, we utilize a vector-autoregressive (VAR) process, following Stock and Watson (2016). As before, $R$ will refer to the total number of factors we extracted in the previous section, and $f^i_t$ for $i = 1, \dots, R$ will refer to the value of factor $i$ at time $t$.

We will use a VAR(1) model of the following form.
\begin{align*}
\underbrace{\begin{bmatrix}
	f^1_{t}\\
	f^2_{t}\\
	\vdots \\
	f^R_{t}
\end{bmatrix}}_{z_t}
=
B
\underbrace{\begin{bmatrix}
	f^1_{t-1}\\
	f^2_{t-1}\\
	\vdots \\
	f^R_{t-1}
\end{bmatrix}}_{z_{t-1}}
+
C
+
\underbrace{\begin{bmatrix}
v^1_t\\
v^2_t\\
\vdots\\
v^R_t
\end{bmatrix}}_{v_t},\\
\text{where $z_t$ is the $R \times 1$ matrix of time $t$ factors,}\\
\text{$B$ is the $R \times R$ coefficient matrix,}\\
\text{$C$ is the $R \times 1$ constant matrix,}\\
\text{and $v_t$ is the $R \times 1$ matrix of errors for time $t$.}
\end{align*}


We wish to estimate the coefficient matrices $B$ and $C$. This can be done via OLS estimation. We first rewrite the data as the standard linear equation,
\begin{align*}
\underbrace{\begin{bmatrix}
f^1_{2} & f^2_{2} & \dots & f^R_{2}\\
f^1_{3} & f^2_{3} & \dots & f^R_{3}\\
\vdots & \vdots & \vdots & \vdots \\
f^1_{T} & f^2_{T} & \dots & f^R_{T}
\end{bmatrix}}_{\Gamma}
=
\underbrace{\begin{bmatrix}
1 & f^1_{1} & f^2_{1} & \dots & f^R_{1}\\
1 & f^1_{2} & f^2_{2} & \dots & f^R_{2}\\
\vdots & \vdots & \vdots & \vdots & \vdots \\
1 & f^1_{T-1} & f^2_{T-1} & \dots & f^R_{T-1}
\end{bmatrix}}_{\Psi}
\underbrace{\begin{bmatrix}
C'\\
B'
\end{bmatrix}}_{\Lambda}
 +
\underbrace{\begin{bmatrix}
v^1_2 & v^2_2 & \dots & v^R_2\\
v^1_3 & v^2_3 & \dots & v^R_3\\
\vdots\\
v^1_T & v^2_T & \dots & v^R_T\\
\end{bmatrix}}_{V},\\
\text{where $\Gamma$ is the $T-1 \times R$ dependent data matrix,}\\
\text{$\Psi$ is the $T-1 \times R+1$ independent data matrix,}\\
\text{$\Lambda$ is the $R+1 \times R$ matrix of coefficient weightings,}\\
\text{and $V$ is the $T-1 \times R$ matrix of residuals.}
\end{align*}
The coefficient matrix $\Lambda$ can be estimated by the standard OLS estimator.
\begin{align*}
\widehat{\Lambda} = (\Psi' \Psi)^{-1} (\Psi'\Gamma)
\end{align*}
It can then be partitioned to calculate $\widehat{B}'$ and $\widehat{C}'$, which can then be transposed to derive our estimates of the original coefficient matrices B and C, $\widehat{B}$ and $\widehat{C}$.


Finally, we perform a qualitative check of the fitted values and residuals. It is important that factors that are predictable --- i.e., factors 2 and 3, since they represent output --- have a good fit. Since factor 1 represents the COVID-19 shock, we should expect that the fit is poor; such a shock should not be predictable simply from the time dynamics of the factors; so if the fit is good, our model is likely overfitted.

\end{document}
